\documentclass[12pt,a4paper,final]{article}
\usepackage[hmargin=2cm,vmargin=2cm,bmargin=2cm]{geometry} % Margem Obrigatoria
\usepackage{pslatex} %times new roman obrigatorio
\usepackage{setspace}
\usepackage[T1]{fontenc}
\usepackage[utf8x]{inputenc}
\usepackage{ucs}
\usepackage{amsmath}
\usepackage{titling}
%\usepackage[•]{•}kage{amsfonts}
\usepackage{amssymb}
\usepackage{graphicx}
%\graphicspath{{images/}}
%\usepackage{float}
%\usepackage{caption}
\usepackage{floatrow}
\usepackage{multirow}
\usepackage{url}
\usepackage[brazil]{babel} % em portugues brasileiro
\usepackage[table]{xcolor}
\usepackage{listings}
\usepackage{color}

\definecolor{codegreen}{rgb}{0,0.6,0}
\definecolor{codegray}{rgb}{0.5,0.5,0.5}
\definecolor{codepurple}{rgb}{0.58,0,0.82}
\definecolor{backcolour}{rgb}{0.95,0.95,0.92}

\lstdefinestyle{mystyle}{
    backgroundcolor=\color{backcolour},
    commentstyle=\color{codegreen},
    keywordstyle=\color{magenta},
    numberstyle=\tiny\color{codegray},
    stringstyle=\color{codepurple},
    basicstyle=\footnotesize,
    breakatwhitespace=false,
    breaklines=true,
    captionpos=b,
    keepspaces=true,
    numbers=left,
    numbersep=5pt,
    showspaces=false,
    showstringspaces=false,
    showtabs=false,
    tabsize=2
}

\lstset{style=mystyle}\usepackage{color}

\definecolor{codegreen}{rgb}{0,0.6,0}
\definecolor{codegray}{rgb}{0.5,0.5,0.5}
\definecolor{codepurple}{rgb}{0.58,0,0.82}
\definecolor{backcolour}{rgb}{0.95,0.95,0.92}

\lstdefinestyle{mystyle}{
    backgroundcolor=\color{backcolour},
    commentstyle=\color{codegreen},
    keywordstyle=\color{magenta},
    numberstyle=\tiny\color{codegray},
    stringstyle=\color{codepurple},
    basicstyle=\footnotesize,
    breakatwhitespace=false,
    breaklines=true,
    captionpos=b,
    keepspaces=true,
    numbers=left,
    numbersep=5pt,
    showspaces=false,
    showstringspaces=false,
    showtabs=false,
    tabsize=2
}

\lstset{style=mystyle}
\renewcommand{\lstlistingname}{Listagem}
\newcommand{\blue}{\textcolor{blue}}
\floatsetup[table]{capposition=top}
\author{Lucas Baganha Galante 182364\\Tiago Loureiro Chaves }

\title{Cálculo Numérico \\ Projeto 1 \\  \normalsize{Turma D}}

\date{ 2 de Outubro de 2018}

\begin{document}

\onehalfspace %Espaçamento 1.5 obrigatório

\maketitle

% \begin{abstract}
% \end{abstract}

\section{Questão 1}
\textbf{Fazer o projeto 1: ``Precisão da Máquina'',
da página 24 do livro \textit{Cálculo Numérico}, Ruggiero-Lopes, 2a. edição.}

\subsection{Parte A}

\begin{lstlisting}
  The single precision is: 0.00000011920928955078
  The double precision is: 0.00000000000000022204
\end{lstlisting}

Como esperado a precisão dupla teve maior precisão que a simples para 20 casas decimais.

\subsection{Parte B}

Devido a presença de números negativos e positivos (não tenho certeza disso então confirmem).

\subsection{Parte C}

\begin{lstlisting}
  The single precision for VAL:    10.0 is: 0.00000095367431640625
  The single precision for VAL:    17.0 is: 0.00000190734863281250
  The single precision for VAL:   100.0 is: 0.00000762939453125000
  The single precision for VAL:   184.0 is: 0.00001525878906250000
  The single precision for VAL:  1000.0 is: 0.00006103515625000000
  The single precision for VAL:  1575.0 is: 0.00012207031250000000
  The single precision for VAL: 10000.0 is: 0.00097656250000000000
  The single precision for VAL: 17893.0 is: 0.00195312500000000000
\end{lstlisting}

Quanto maior VAL, menor será a precisão de máquina pois serão necessários
mais dígitos da mantissa para representar s, o que diminui o número
de dígitos que podem ser usados para determinar a precisão.

\subsection{Código}

\lstinputlisting[language=C]{../exercicio1/exercicio1.c}

\section{Questão 2}

\subsection{Parte A}

\textbf{Como provar isso?}

\subsection{Parte B}

\begin{lstlisting}
  Media: 4.5 F_m: -107.1875
  Media: 6.75 F_m: 3160.56835938
  Media: 5.625 F_m: 448.283996582
  Media: 5.0625 F_m: 25.666475296
  Media: 4.78125 F_m: -65.7846035361
  Media: 4.921875 F_m: -27.5399343763
  Media: 4.9921875 F_m: -2.97468937194
  Media: 5.02734375 F_m: 10.8144866157
  Media: 5.009765625 F_m: 3.78982958678
  Media: 5.0009765625 F_m: 0.375396885005
  Media: 4.99658203125 F_m: -1.30764677991
  Media: 4.99877929688 F_m: -0.468130417219
  Media: 4.99987792969 F_m: -0.0468688014225
  Media: 5.00042724609 F_m: 0.164138449422
  Media: 5.00015258789 F_m: 0.0586034363523
  Media: 5.00001525879 F_m: 0.00585947185755
  Media: 4.99994659424 F_m: -0.0205066260205
  Media: 4.99998092651 F_m: -0.0073240674119
  Media: 4.99999809265 F_m: -0.000732420361601
  Media: 5.00000667572 F_m: 0.00256349510164
  Media: 5.00000238419 F_m: 0.000915529709346
  Media: 5.00000023842 F_m: 9.15527575671e-05
  METODO DA BISSECAO
  Achou a raiz: 5.00000023842
  Numero de iteracoes: 22
  -------------------------------
  Media: 0.00627352572146 F_m: -19.5398354788
  Media: 0.272664468156 F_m: -6.28387508792
  Media: 0.398944824875 F_m: -3.19916587126
  Media: 0.529910753606 F_m: -1.36530864458
  Media: 0.627415026044 F_m: -0.620845449741
  Media: 0.708728762294 F_m: -0.273858511139
  Media: 0.772905410655 F_m: -0.121498345517
  Media: 0.824082544553 F_m: -0.0534669661609
  Media: 0.864303485423 F_m: -0.0234718342023
  Media: 0.895777234458 F_m: -0.0102613711505
  Media: 0.92022481204 F_m: -0.00447304307117
  Media: 0.939117152973 F_m: -0.00194447669548
  Media: 0.953645431535 F_m: -0.00084343269782
  Media: 0.96477452829 F_m: -0.000365176821177
  Media: 0.973272255917 F_m: -0.000157879038625
  Media: 0.979744168215 F_m: -6.81778053888e-05
  METODO DA SECANTE
  Achou a raiz: 0.979744168215
  Numero de iteracoes: 16
  -------------------------------
  Media: 0.645161290323 F_m: -0.527208787994
  Media: 0.751292823997 F_m: -0.163234649909
  Media: 0.827637023342 F_m: -0.050096355305
  Media: 0.881697563505 F_m: -0.0152506576222
  Media: 0.919439444395 F_m: -0.00461069390615
  Media: 0.945474617282 F_m: -0.00138619147897
  Media: 0.963263407207 F_m: -0.000414978367587
  Media: 0.97532981075 F_m: -0.000123840254886
  Media: 0.983471233981 F_m: -3.68741353824e-05
  METODO DE NEWTON
  Achou a raiz: 0.983471233981
  Numero de iteracoes: 9
  -------------------------------
  Media: 1.05138339921 F_m: 0.00101632866931
  Media: 1.00120254996 F_m: 1.38914231229e-08
  METODO DE NEWTON PARA RAIZES MULTIPLAS
  Achou a raiz: 1.00120254996
  Numero de iteracoes: 2
  -------------------------------
\end{lstlisting}

\subsection{Código}

\lstinputlisting[language=Python]{../exercicio2/projeto1-2.py}


\section{Questão 3}




\bibliographystyle{IEEEtran}
\footnotesize{
\bibliography{}
}
\end{document}
